\chapter{Решаем задачи в функциональном стиле}
\label{functionally-solving-problems}
\colorbox{lgray}
{
\begin{minipage}{1.0\linewidth}
    Похоже, что мы уже достаточно напились эрлангового сока, чтобы сделать что\--нибудь полезное.
    В этой главе не будет ничего нового, а просто будет показано как применять элементы увиденного ранее.
    Задачи были взяты из книги Miran\--а \href{http://learnyouahaskell.com/functionally-solving-problems}{Learn You a Haskell}.
    Я использовал те же самые пути решения для того, чтобы любопытный читатель смог сравнивать решения на Erlang и Haskell как ему заблагорассудится.
    Если вы сделаете такое сравнение, то наверняка найдёте, что для двух языков с такими разными синтаксисами, конечные результаты очень похожи.
    Так происходит потому, что после изучения концепций функционального программирования, их относительно легко переносить на другие функциональные языки.
\end{minipage}
}
