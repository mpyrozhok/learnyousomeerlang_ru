\chapter{Ошибки и процессы}
\label{errors-and-processes}
\section{Связи}
\label{links}
Связи \--- это особый вид взаимоотношений, которые могут быть установлены между двумя процессами.
Когда один из процессов, который участвует в таких отношениях, умирает от неожиданного броска, ошибки или завершения (см. \ref{errors-and-exceptions} Ошибки и исключения), то связанный с ним процесс тоже завершается.

Эта концепция может пригодиться, когда требуется как можно быстрее завершить процесс, чтобы предотвратить появление ошибок.
Если процесс, в котором появилась ошибка, завершился аварией, а процессы, которые на него полагаются, продолжили работать, то все эти зависимые процессы должны что\--то предпринять.
Обычно приемлемым вариантом развития событий можно считать остановку и перезапуск всей группы процессов.
Именно это и позволяют нам сделать связи.
