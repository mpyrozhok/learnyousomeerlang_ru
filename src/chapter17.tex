\chapter{Клиенты и серверы}
\label{clients-and-servers}
\section{Обратный вызов в будущее}
\label{callback-to-the-future}
\begin{wrapfigure}{l}{0.35\linewidth}
    \includegraphics[width=1\linewidth]{cbttf.png}
\end{wrapfigure}
Первым шаблоном поведения, который мы рассмотрим, будет самый часто используемый.
Его имя \ops{gen\_server} и его интерфейс немного схож с тем, что мы записали для \ops{my\_server} в предыдущей главе \ref{the-basic-server}. Он предоставляет вам в пользование несколько функций, а взамен ваш модуль должен содержать несколько функций, которыми воспользуется \ops{gen\_server}.
\subsection{Инициализация}
\label{init}
Первая функция \--- \href{http://erldocs.com/R15B/stdlib/gen\_server.html#init/1}{init/1}.
Она похожа на функцию, которую мы создавали  для \ops{my\_server}, тем, что её можно использовать для инициализации состояния сервера, и исполнения всех одноразовых задач, от которых эта инициализация будет зависеть.
Функция может возвращать \ops{\{ok, State\}}, \ops{\{ok, State, TimeOut\}}, \ops{\{ok, State, hibernate\}}, \ops{\{stop, Reason\}} или \ops{ignore}.

Значение \ops{\{ok, State\}}, которое возвращается в случае успеха, в подробном объяснении не нуждается.
Нужно лишь сказать, что \emph{State} будет передан прямо в главный цикл процесса, в качестве состояния, которое необходимо сохранить и использовать позже.
Переменную \emph{TimeOut} нужно добавлять в кортеж, когда вам необходимо установить конечный срок, в течение которого вы ожидаете, что сервер получит сообщение.
Если до указанного момента сообщение не получено, серверу высылается особое сообщение (атом \ops{timeout}), который должен быть обработан функцией \href{http://erldocs.com/R15B/stdlib/gen\_server.html#handle\_info/2}{handle\_info/2} (об этом чуть позже).

Но если вы, напротив, ожидаете, что пройдёт много времени перед тем, как придёт ответ, и переживаете, что на протяжении всего временного отрезка память будет занята без толку, вы можете добавить в кортеж атом \ops{hibernate}.
Хибернация, по сути, уменьшает размер состояния процесса до получения сообщения, жертвуя при этом некоторым количеством вычислительных ресурсов.
Если вы не уверены, стоит ли вам использовать хибернацию, то вам она, скорее всего ни к чему.

Кортеж \ops{\{stop, Reason\}} нужно возвращать, когда во время инициализации что\--то пошло не по плану.

\colorbox{lgray}
{
\begin{minipage}{1.0\linewidth}
    \textbf{Замечание:} вот более техническое определение хибернации процессов.
    Нет ничего страшного в том, что кто\--то из читателей его не поймёт, или не найдёт его полезным.
    Когда вызывается встроеная функция \href{http://erldocs.com/R15B/erts/erlang.html\#hibernate/3}{erlang:hibernate(M,F,A)}, стек вызовов для исполняемого процесса выбрасывается (функция никогда не завершается).
    Запускается сборка мусора, и после её окончания остаётся одно непрерывное пространство кучи (heap), уменьшенное до размера данных, которые использует процесс.
    По сути, эта процедура просто уплотняет все данные, чтобы уменьшить пространство, которое занимает процесс.

    Как только процесс получит сообщение, будет вызвана функция \ops{M:F} с \emph{A} в качестве аргумента, и процесс продолжит выполняться.
\end{minipage}
}

\colorbox{lgray}
{
\begin{minipage}{1.0\linewidth}
    \textbf{Замечание:} на то время, пока работает функция \ops{init/1}, блокируется процесс, выполняющий запуск сервера.
    Этот простой возникает из\--за того, что процесс ожидает сообщение 'ready', которое автоматически высылается модулем \ops{gen\_server}, как подтверждение того, что всё прошло нормально.
\end{minipage}
}

\subsection{handle\_call}
\label{handle-call}
Функция \href{http://erldocs.com/R15B/stdlib/gen\_server.html#handle\_call/3}{handle\_call/3} используется для работы с синхронными сообщениями (мы скоро увидим, как их можно отсылать).
Она принимает 3 аргумента: \emph{Request}, \emph{From} и \emph{State}.
Очень похоже на то, как мы запрограммировали нашу собственную функцию \ops{handle\_call/3} в \ops{my\_server}.
Самое большое отличие заключается в том, как мы отвечаем на сообщения.
В нашем собственноручно изготовленном обобщённом сервере, для общения с процессом нужно было использовать \ops{my\_server:reply/2}.
А для \ops{gen\_server} можно возвращать 8 различных значений, которые заключаются в кортежи.
Сообщений этих немало, поэтому я приведу здесь их перечень:
\begin{lstlisting}[style=erlang]
{reply,Reply,NewState}
{reply,Reply,NewState,Timeout}
{reply,Reply,NewState,hibernate}
{noreply,NewState}
{noreply,NewState,Timeout}
{noreply,NewState,hibernate}
{stop,Reason,Reply,NewState}
{stop,Reason,NewState}
\end{lstlisting}
Для всех этих сообщений \emph{Timeout} и \ops{hibernate} будут выполнять те же функции, что и для \ops{init/1}.
Что бы ни находилось в \emph{Reply}, его содержимое будет отослано тому, кто совершил вызов к серверу.
Обратите внимание, что существует три возможных варианта \ops{noreply}.
Когда вы используете \ops{noreply}, общая часть сервера будет предполагать, что вы сами позаботитесь об отсылке ответного сообщения.
Эту операцию можно совершить при помощи вызова \href{http://erldocs.com/R15B/stdlib/gen\_server.html#reply/2}{gen\_server:reply/2}, который можно использовать в том же порядке, что и \ops{my\_server:reply/2}.

В большинстве случаев вы будете использовать только кортежи \ops{reply}.
Но для использования \ops{noreply} всё же можно найти несколько веских причин.
Его стоит использовать, если вы хотите отослать ответ из другого процесса, или если вам нужно отослать подтверждение (<<Эй! Я получил сообщение!>>), но отложить обработку принятого собщения на потом (на этот раз уже без отсылки ответа), и т.д.
Если вам необходимо совершить такие действия, то вам точно придётся использовать \ops{gen\_server:reply/2}, так как при использовании другой функции вызов отвалится по тайм\--ауту и станет причиной аварийной ситуации. 
\subsection{handle\_cast}
\label{handle-cast}
Принцип работы функции обратного вызова (callback) \href{http://erldocs.com/R15B/stdlib/gen\_server.html#handle\_cast/2}{handle\_cast/2} очень похож на одну из функций, написанных нами для \ops{my\_server}.
Она принимает в качестве параметров \emph{Message} и \emph{State}, и используется для обработки асинхронных вызовов.
В ней можно делать что угодно, и в этом её поведение весьма похоже на \ops{handle\_call/3}.
Но есть одно отличие: из этой функции можно возвращать только кортежи без ответного сообщения:
\begin{lstlisting}[style=erlang]
{noreply,NewState}
{noreply,NewState,Timeout}
{noreply,NewState,hibernate}
{stop,Reason,NewState}
\end{lstlisting}
\subsection{handle\_info}
\label{hande-info}
Помните, я упоминал о том, что наш сервер игнорировал сообщения, не соответствующие нашему интерфейсу?
Так вот, вызов \href{http://erldocs.com/R15B/stdlib/gen\_server.html#handle\_info/2}{handle\_info/2} решает эту проблему.
Он очень похож на функцию \ops{handle\_cast/2} и возвращает кортежи того же вида.
Отличие состоит в том, что эта функция обратного вызова работает только для сообщений, посланных напрямую при помощи оператора \ops{!} и особенных функций, таких как \ops{timeout} в \ops{init/1}, уведомлений мониторов и сигналов \ops{'EXIT'}.
\subsection{terminate}
\label{terminate}
Функция обратного вызова \href{http://erldocs.com/R15B/stdlib/gen\_server.html#terminate/2}{terminate/2} вызывается, если одна из трёх функций вида \ops{handle\_Нечто} возвращает кортеж вида \ops{\{stop, Reason, NewState\}}, или \ops{\{stop, Reason, Reply, NewState\}}.
Она принимает два параметра: \emph{Reason} и \emph{State}, которые соответствуют тем же значениям из кортежей \ops{stop}.

Функция \ops{terminate/2} также будет вызываться, когда родитель (процесс, который её породил), умирает, но лишь в том случае, если \ops{gen\_server} захватывает завершения (trapping exits).

\colorbox{lgray}
{
\begin{minipage}{1.0\linewidth}
    \textbf{Замечание:} если для вызова \ops{terminate/2} используется причина, не входящая в число \ops{normal}, \ops{shutdown} или \ops{\{shutdown, Term\}}, то фреймворк OTP расценит этот вызов как аварийную ситуацию и сделает об этом несколько записей в журнале.
\end{minipage}
}

Эта функция \--- прямая противоположность \ops{init/1}, поэтому в \ops{terminate/2} должно содержаться обратное действие для каждого действия из \ops{init/1}.
Эта функция выполняет для вашего сервера функции уборщика, который запирает двери, убедившись, что все ушли. 
Конечно, этой функции помогает сама виртуальная машина, которая обычно удаляет за вас все \href{http://erldocs.com/R15B/stdlib/ets.html}{ETS таблицы}, закрывает все \href{http://www.erlang.org/doc/tutorial/c\_port.html}{порты} и т.д.
Заметьте, что значение, возвращаемое функцией, не так уж и важно, так как после её вызова останавливается выполнение кода.
\subsection{code\_change}
\label{code-change}
Функция \href{http://erldocs.com/R15B/stdlib/gen\_server.html#code\_change/3}{code\_change/3} позволяет вам производить модернизацию (upgrade) кода.
Её сигнатура имеет вид \ops{code\_change(PreviousVersion, State, Extra)}.
Переменная \emph{PreviousVersion} содержит терм, описывающий версию, если производится модернизация (если вы не помните, что это такое, перечитайте раздел \ref{more-about-modules}), или кортеж \ops{\{down, Version\}}, если производится понижение версии (downgrade) (загрузка более старого кода).
Переменная \emph{State} содержит всё текущее состояние сервера, готовое для приведения к виду, подходящему для новой версии кода.

Представьте, что для хранения всех наших данных мы использовали orddict.
Но с течением времени orddict сильно замедляется, и мы решаем  поменять его на обычный dict.
Чтобы избежать краха процесса при следующем вызове функции, мы можем безопасно произвести преобразование структур данных.
Для этого лишь нужно вернуть новое состояние в кортеже \ops{\{ok, NewState\}}.
\begin{wrapfigure}{r}{0.35\linewidth}
    \includegraphics[width=1\linewidth]{kitty.png}
\end{wrapfigure}

О переменной \emph{Extra} мы пока переживать не будем.
По большей части она используется при более масштабных развёртываниях OTP приложений, в которых используются специальные инструменты, позволяющие производить в виртуальной машине полную модернизацию релизов.
Но мы пока к этому не готовы.

Итак, мы определили все обратные вызовы.
Не переживайте, если вы немного запутались: в фреймворке OTP иногда приходится ходить кругами.
Чтобы понять раздел \emph{A}, вы должны понимать раздел \emph{B}, но для успешного использования раздела \emph{B} необходимо иметь понятие о разделе \emph{A}. 
Выйти из этого состояния замешательства проще всего, реализовав gen\_server своими руками.
\section{\href{https://en.wikipedia.org/wiki/Beam\_me\_up,\_Scotty}{.BEAM me up, Scotty!}}
\label{beam-me-up-scotty}
Сейчас мы займёмся \href{http://learnyousomeerlang.com/static/erlang/kitty_gen_server.erl}{kitty\_gen\_server}.
В основном он будет похож на \ops{kitty\_server2}, с минимальными изменениями в API.
Для начала создайте новый модуль, и поместите в него следующие строки:
\begin{lstlisting}[style=erlang]
-module(kitty_gen_server).
-behaviour(gen_server).
\end{lstlisting}
Теперь попробуйте его скомпилировать.
У вас должно получиться что\--то вроде этого:
\begin{lstlisting}[style=erlang]
1> c(kitty_gen_server).
./kitty_gen_server.erl:2: Warning: undefined callback function code_change/3 (behaviour 'gen_server')
./kitty_gen_server.erl:2: Warning: undefined callback function handle_call/3 (behaviour 'gen_server')
./kitty_gen_server.erl:2: Warning: undefined callback function handle_cast/2 (behaviour 'gen_server')
./kitty_gen_server.erl:2: Warning: undefined callback function handle_info/2 (behaviour 'gen_server')
./kitty_gen_server.erl:2: Warning: undefined callback function init/1 (behaviour 'gen_server')
./kitty_gen_server.erl:2: Warning: undefined callback function terminate/2 (behaviour 'gen_server')
{ok,kitty_gen_server}
\end{lstlisting}

Компиляция прошла успешно, но мы получили предупреждения о пропущенных функциях обратного вызова.
Возникли они из\--за стратегии поведения \ops{gen\_server}.
Стратегия поведения, по сути, предоставляет модулю возможность указать те функции, которые он ожидает увидеть в другом модуле.
Стратегия поведения \--- это контракт, заключённый между корректными частями кода и участками, которые могут содержать ошибки (вашим кодом).

\colorbox{lgray}
{
\begin{minipage}{1.0\linewidth}
    \textbf{Замечание:} обратите внимание, что компилятор Erlang допускает оба варианта написания: 'behavior' и 'behaviour'.
\end{minipage}
}

Добавлять собственные стратегии поведения очень просто.
Нужно лишь проэкспортировать функцию с именем \ops{behaviour\_info/1}, реализованную следующим образом:
\begin{lstlisting}[style=erlang]
-module(my_behaviour).
-export([behaviour_info/1]).
 
%% init/1, some_fun/0 and other/3 are now expected callbacks
behaviour_info(callbacks) -> [{init,1}, {some_fun, 0}, {other, 3}];
behaviour_info(_) -> undefined.
\end{lstlisting}

Вот, пожалуй, и всё, что можно сказать о стратегиях поведения.
В модуле, который их будет реализовать, достаточно добавить строку \ops{-behaviour(my\_behaviour)}, чтобы получить предупреждение компилятора в случае, если вы забыли реализовать функцию.
Вернёмся к третьей инкарнации нашего котосервера.

Первой реализованной нами функцией была \ops{start\_link/0}.
Её можно привести к следующему виду:
\begin{lstlisting}[style=erlang]
start_link() -> gen_server:start_link(?MODULE, [], []).
\end{lstlisting}

Первый параметр задаёт модуль обратного вызова, второй параметр содержит список параметров, которые нужно передать функции \ops{init/1}, а третий параметр определяет параметры отладки, которых мы сейчас касаться не будем.
Также можно добавить первым по порядку \href{http://erldocs.com/R15B/stdlib/gen_server.html#start_link/4}{четвёртый параметр}, определяющий имя, под которым будет зарегистрирован сервер.
Отметьте, что предыдущая версия функции просто возвращала pid, а эта возвращает кортеж \ops{\{ok, Pid\}}.
А теперь остальные функции:
\begin{lstlisting}[style=erlang]

%% Synchronous call
order_cat(Pid, Name, Color, Description) ->
    gen_server:call(Pid, {order, Name, Color, Description}).
 
%% This call is asynchronous
return_cat(Pid, Cat = #cat{}) ->
    gen_server:cast(Pid, {return, Cat}).
 
%% Synchronous call
close_shop(Pid) ->
    gen_server:call(Pid, terminate).
\end{lstlisting}

Все эти функции перенесены один к одному.
Обратите внимание, что \href{http://erldocs.com/R15B/stdlib/gen\_server.html#call/2}{gen\_server:call/2-3} можно передать третий параметр, который задаст величину тайм\--аута.
Если тайм\--аут не указан (или указан атом \ops{infinity}), то по умолчанию задаётся значение равное 5 секундам.
Если до истечения интервала не получен ответ, вызов завершится аварией.
Теперь мы сможем добавить функции обратного вызова для gen\_server.
Следующая таблица показывает взаимосвязь между вызовами и функциями обратного вызова:
\begin{tabular}{l|l}
    \textbf{gen\_server} & \textbf{YourModule} \\
    \hline
    \ops{start/3-4} & \ops{init/1} \\
    \ops{start\_link/3-4} & \ops{init/1} \\
    \ops{call/2-3} & \ops{handle\_call/3} \\
    \ops{cast/2} & \ops{handle\_cast/2} \\
\end{tabular}

Ну и остаются ещё другие функции обратного вызова, которые относятся к особенным случаям:
\begin{itemize}
    \item \ops{handle\_info/2}
    \item \ops{terminate/2}
    \item \ops{code\_change/3}
\end{itemize}

Давайте изменим функции \ops{init/1}, \ops{handle\_call/3} и \ops{handle\_cast/2}, так, чтобы они соответствовали нашей модели.
\begin{lstlisting}[style=erlang]
%%% Server functions
init([]) -> {ok, []}. %% no treatment of info here!
 
handle_call({order, Name, Color, Description}, _From, Cats) ->
    if Cats =:= [] ->
        {reply, make_cat(Name, Color, Description), Cats};
        Cats =/= [] ->
        {reply, hd(Cats), tl(Cats)}
    end;
handle_call(terminate, _From, Cats) ->
    {stop, normal, ok, Cats}.
 
handle_cast({return, Cat = #cat{}}, Cats) ->
    {noreply, [Cat|Cats]}.
\end{lstlisting}

И снова изменений получилось не так уж и много.
Код даже стал короче, благодаря толковым абстракциям.
Теперь обратимся к новым функциям обратного вызова.
Первая на очереди \--- \ops{handle\_info/2}.
Так как этот модуль серьёзным назвать нельзя, и мы не используем в нём систему журналирования, будет достаточно просто вывести непредусмотренные (unexpected) сообщения на печать:
\begin{lstlisting}[style=erlang]
handle_info(Msg, Cats) ->
    io:format("Unexpected message: ~p~n",[Msg]),
    {noreply, Cats}.
\end{lstlisting}

Следующая на очереди \--- функция \ops{terminate/2}.
Она будет очень похожа на закрытую (private) функцию \ops{terminate/1}, написанную нами ранее:
\begin{lstlisting}[style=erlang]
terminate(normal, Cats) ->
    [io:format("~p was set free.~n",[C#cat.name]) || C <- Cats],
    ok.
\end{lstlisting}
Ну и последняя функция \--- \ops{code\_change/3}:
\begin{lstlisting}[style=erlang]
code_change(_OldVsn, State, _Extra) ->
    %% No change planned. The function is there for the behaviour,
    %% but will not be used. Only a version on the next
    {ok, State}.
\end{lstlisting}
Не забудьте перенести закрытую (private) функцию \ops{make\_cat/3}:
\begin{lstlisting}[style=erlang]
%%% Private functions
make_cat(Name, Col, Desc) ->
    #cat{name=Name, color=Col, description=Desc}.
\end{lstlisting}
Теперь мы можем опробовать наш новенький код в деле:
\begin{lstlisting}[style=erlang]
1> c(kitty_gen_server).
{ok,kitty_gen_server}
2> rr(kitty_gen_server).
[cat]
3> {ok, Pid} = kitty_gen_server:start_link().
{ok,<0.253.0>}
4> Pid ! <<"Test handle_info">>.
Unexpected message: <<"Test handle_info">>
<<"Test handle_info">>
5> Cat = kitty_gen_server:order_cat(Pid, "Cat Stevens", white, "not actually a cat").
#cat{name = "Cat Stevens",color = white,
    description = "not actually a cat"}
6> kitty_gen_server:return_cat(Pid, Cat).
ok
7> kitty_gen_server:order_cat(Pid, "Kitten Mittens", black, "look at them little paws!").
#cat{name = "Cat Stevens",color = white,
    description = "not actually a cat"}
8> kitty_gen_server:order_cat(Pid, "Kitten Mittens", black, "look at them little paws!").
#cat{name = "Kitten Mittens",color = black,
    description = "look at them little paws!"}
9> kitty_gen_server:return_cat(Pid, Cat).
ok      
10> kitty_gen_server:close_shop(Pid).
"Cat Stevens" was set free.
ok
\end{lstlisting}
И он работает, чёрт возьми!

\begin{wrapfigure}{r}{0.35\linewidth}
    \includegraphics[width=1\linewidth]{mittens.png}
\end{wrapfigure}
Так что же мы можем извлечь из этого обобщённого приключения?
Наверно, приблизительно то же, что и раньше: разделение общего и частного это прекрасная идея во всех смыслах.
Поддержка кода упрощается, сложность уменьшается, код становится безопаснее, его проще тестировать и он менее подвержен ошибкам.
А если ошибки в нём всё\--таки есть, то их проще устранять.
Обобщённый сервер это лишь одна абстракция из множества доступных, но она, безусловно, относится к наиболее используемым.
В следующей главе мы рассмотрим ещё больше абстракций и стратегий поведения.
